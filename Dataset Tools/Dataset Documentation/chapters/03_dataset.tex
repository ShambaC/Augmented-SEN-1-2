The dataset format remains unchanged. The pairs of images are divided between 4 seasons and are grouped by regions into specific folders. However, the data acquiring process has been changed slightly. On top using Google Earth Engine\cite{GORELICK201718} we also use the Copernicus Data Space Ecosystem API to access Sentinel data.

\subsection{Image format}
The images are in png format with the dimension of $256\times256$ pixels. The scale or zoom level of the image in terms of distance is 20m. Now, the sentinel satellite specific image details will be defined in the following points:
\begin{itemize}
    \item \textbf{Sentinel-1:} The images are taken from the IW acquisition mode of the satellite and are in VV polarization. It comprises of only a single band.
    \item \textbf{Sentinel-2:} The images consist of 3 bands for Red, Green and Blue. The bands are also scaled for proper visualisation.
\end{itemize}

\subsection{Copernicus API Method}
This section contains an overview of how the copernicus API was used to obtain the images.